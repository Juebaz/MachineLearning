\documentclass{article}
\usepackage[utf8]{inputenc}
\usepackage{amsmath}

\title{devoir3Q1}
\author{jujubazzz95 }
\date{November 2021}

\begin{document}

\maketitle

\section{Question1}

\subsection{La couche de sortie}

Pour trouver les poids, on dérive l'erreur quadratique par rapport au poids $ w_{j,i}$
$$
\Delta w_{j,i} = -\eta \frac{\partial E}{\partial w_{j,i}}= \frac{-\eta}{N} \sum_{t=1}^{N} \frac{\partial E^t}{\partial w_{j,i}}
$$

La dérivé peut être exprimer comme :

\begin{align}
    \frac{\partial E^t}{\partial w_{j,i}}=\frac{\partial E^t}{\partial e_{j}^t}\frac{\partial e_{j}^t}{\partial p(z)_{j}^t}\frac{\partial p(z)_{j}^t}{\partial z_{j}^t}\frac{\partial z_{j}^t}{\partial w_{j,i}}
\end{align}


\begin{align}
\frac{\partial E^t}{\partial e_{j}^t}=\frac{\partial}{\partial e_{j}^t}\Big[\frac{1}{2}\sum_{l=1}^N(e_l^t)^2\Big] = e_j^t
\end{align}

\begin{align}
\frac{\partial e_j^t}{\partial p_{j}^t}=\frac{\partial}{\partial p_j^t}(r_i^t-y_i^t)=-1
\end{align}

On considère ici que $p(z)$ est $p(z_j^t)$ considérant donc la somme pondérée des poids
\begin{align}
\frac{\partial p_j^t}{\partial z_{j}^t}=\frac{\partial}{\partial z_j^t}\Bigg[\frac{\exp{(2z_j^t)-1}}{\exp(2z_j^t)+1}\Bigg]=\frac{\partial}{\partial z_j^t} \tanh(z_{j}^t)=sech(z_{j}^t)^2
\end{align}

\begin{align}
\frac{\partial z_j^t}{\partial w_{j,i}}=\frac{\partial}{\partial w_{j,i}}\sum_{k=1}^R
 w_{j,i}y_k^t+w_{j,0}=p_i^t
\end{align}

\begin{align}
\frac{\partial z_j^t}{\partial w_{j,0}}=\frac{\partial}{\partial w_{j,i}}\sum_{k=1}^R
 w_{j,i}y_k^t+w_{j,0}=1
\end{align}
 
En prenant les résultats suivant et en les mettant dans l'équation 1 on obtient 
\begin{align}
&\Delta w_{j,i}=\frac{\eta}{N} \sum_{t=1}^N e_j^t(sech(z_{j}^t)^2)p_i^t 
&\Delta w_{j,0}=\frac{\eta}{N} \times \sum_{t=1}^N e_j^t(sech(z_{j}^t)^2)
\end{align}

\subsection{La couche cachée}

On trouves les poids des neurones de la couche cachée en prenant encore une fois la dérivée partielle de l'erreur quadratique en fonction des poids

\begin{align}
\frac{\partial E^t}{\partial w_{j,i}}=\frac{\partial E^t}{\partial p_{j}^t}\frac{\partial p_{j}^t}{\partial z_{j}^t}\frac{\partial z_{j}^t}{\partial w_{j,i}}
\end{align}


On a déja calculé $\frac{\partial p_{j}^t}{\partial z_{j}^t}$ et $\frac{\partial z_{j}^t}{\partial w_{j,i}}$

Calculons $\frac{\partial E^t}{\partial p_{j}^t}$

\begin{align}
    \frac{\partial E^t}{\partial p_{j}^t} &= \frac{\partial}{\partial p_{j}^t} \Big[\frac{1}{2}\sum_{l=1}^N(e_l^t)^2\Big]\\
    &=  \Big[\frac{2}{2}\sum_{l=1}^N (e_l^t)\frac{\partial}{\partial p_{j}^t}(e_l^t)\Big]\\
    &= \Big[\sum_{l=1}^N(e_l^t)\frac{\partial}{\partial p_{j}^t}(r_j-p^t_j)\Big]\\
    &= \sum_{l=1}^N -e^t_k
    &=
\end{align}

En mettant tout ensemble on obtient: 

\begin{align}
\Delta w_{j,i} &= \frac{\eta}{N}  \sum_{k=1}^K e_k^t(sech(z_j^t)^2)p_i^t \\
\Delta w_{j,0} &= \frac{\eta}{N}  \sum_{t=1}^N e_j^t(sech(z_j^t)^2)
\end{align}


\end{document}
